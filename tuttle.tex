\documentclass{beamer}
%\usetheme{Dresden}

\usepackage[T1]{fontenc} % Use 8-bit encoding that has 256 glyphs
%\usepackage{fourier} % Use the Adobe Utopia font for the document - comment this line to return to the LaTeX default
\usepackage[english]{babel} % English language/hyphenation
\usepackage{amsmath,amsfonts,amsthm} % Math packages
\usepackage{hyperref}
\usepackage{lipsum} % Used for inserting dummy 'Lorem ipsum' text into the template
%\usepackage{sectsty} % Allows customizing section commands 
\usepackage[utf8x]{inputenc}
\usepackage{fancyhdr} % Custom headers and footers
\usepackage{booktabs}
\usepackage{tikz}
\usepackage{amsmath}
\usepackage{amssymb}
\usepackage{relsize}

% Macros for nicely rounded letters needed with all these sets...
\newcommand{\Drr}{\mathcal{D}}
\newcommand{\Krr}{\mathcal{K}}
\newcommand{\Lrr}{\mathcal{L}}
\newcommand{\Prr}{\mathcal{P}}

% Can't find the shorthand for integers
\newcommand{\Nzz}{\mathbb{N}}
\newcommand{\Zzz}{\mathbb{Z}}


%Information to be included in the title page:
\title{Tutte's Characterization of TUMs\\A demonstration by A M Gerards}
\author{Cojocaru Andrei, Pavel Nicolae}
\date{\normalsize\today}
 
\setbeamertemplate{items}[square]
 
\begin{document}

\section{Title}
\frametitle{Latex is an Art Form}
\frame{\titlepage}

\section{Whatever, these section names don't show anywhere}
\begin{frame}
\frametitle{Tutte's Theorem}
\begin{definition}
Let A be a {0, 1}-matrix. Then the following are equivalent:
\begin{enumerate}
\item A has a totally unimodular signing
\item A cannot be transformed to
\begin{center}
M(F$_7$) = $\begin{bmatrix} 1 & 1 &1 & 0 \\ 1 & 1 & 0 & 1 \\ 1 & 0 &1 & 1\end{bmatrix}$
\end{center}
\end{enumerate}
by applying (repeatedly) the following operations:
\begin{equation}
\begin{aligned}
% I am SO happy about learning LaTeX
deleting\, rows\, or\, columns\\
permuting\, rows\, or\, columns\\
taking\, the\, transposed\, matrix\\
pivoting\, over\, GF(2)
\end{aligned}
\end{equation}
\end{definition}
\end{frame}

\begin{frame}
\frametitle{Tutte's Theorem, more definitions}
\begin{definition}
A matrix is \emph{totally unimodular} if all of its subdeterminants are 0, 1 or -1. (In particular, all entries of a totally unimodular matrix are 0, 1 or -1).
\end{definition}
\begin{definition}
Signing a {0, 1}-matrix means replacing some of the 1s by -1s.
\end{definition}
\begin{definition}
Pivoting a matrix A on an entry $\epsilon = \pm 1$ means replacing
\begin{equation}
A = \begin{pmatrix}
\epsilon & y^T \\ x & D
\end{pmatrix}
by\, B = \begin{pmatrix}
-\epsilon & y ^T \\ x & D - \epsilon x y^T
\end{pmatrix}
\end{equation}
\end{definition}
\end{frame}

\begin{frame}
\frametitle{Preliminaries}
\begin{block}{Lemma 1}
Let G be a connected bipartite graph (with no parallel edges). If deleting any two nodes in the same partition yields a disconnected graph, then G is a path or a circuit. 
\end{block}
\begin{block}{Lemma 2}
Let A be a square {0, 1, -1} matrix. If G(A) is a circuit, then A is totally unimodular if and only if the number of -1s in A is congruent to n modulo 2.
\end{block}
\begin{block}{Lemma 3}
Let $M_1$ and $M_2$ be two totally unimodular matrices. If $M_1 \equiv M_2 (mod 2)$, then $M_1$ can be obtained from $M_2$ by multiplying some rows and columns with -1. ($M_1 \equiv M_2 (mod 2): \exists P$ so that $P^T M_1 P = M_2$).
\end{block}
\end{frame}


\begin{frame}
\frametitle{Pivoting properties}
Some useful properties of the pivoting operation (2):
\begin{gather}
\begin{aligned}
\text{(i) Pivoting B on -}\epsilon\text{ yields A}\\
\text{(ii) If A is square, then det(A) = }\pm det(D - \epsilon x y^T)\\
\text{(iii) If A is totally unimodular, then B is totally unimodular}\\
\text{(iv) If G(A) is connected, then G(B) is connected}\\
\text{(G(B) disconnected plus (i)}\implies\text{G(A) disconnected)}
\end{aligned}
\end{gather}
\end{frame}

\begin{frame}
\frametitle{Proof of Tutte's theorem, $\implies$}
\begin{proof}
The first three operations of (1) obviously don't change the matrix having an unimodular signing or not.\par
Property (3) (iii) of pivoting means pivoting does not change the matrix having an unimodular signing either.\par
Since $M(F_7)$ has no totally unimodular signing, (i) $\implies$ (ii).
\end{proof}
\end{frame}

\begin{frame}{allowframebreaks}
\frametitle{Proof of Tutte's theorem, $\impliedby$}
Let A be a {0, 1} matrix, satisfying (ii) but with no totally unimodular signing. We may assume that each proper sumbatrix of A has a totally unimodular signing, so the bipartite graph G(A) is connected.\par
G(A) is not a path or circuit, or A would have a totally unimodular signing. Hence by Lemma 1, A or $A^T$ is equal (up to permutation of columns) to [x|y|N], where x and y are two column vectors and where G(N) is connected. (x and y correspond to the nodes we eliminate and still get a connected graph).\par
By the initial assumption, both [x|N] and [y|N] have a totally unimodular signing, and by \emph{Lemma 3} they can be chosen so that N is signed in the same way in both cases.\par
Hence A or $A^T$ has a signing A'=[x'|y'|N']\par
\begin{gather}
\begin{aligned}
\text{(i) G(N') is connected}\\
\text{(ii) Both [x'|N'] and [y'|N'] are totally unimodular}
\end{aligned}
\end{gather}
\end{frame}

\begin{frame}
\frametitle{Proof of Tutte's theorem, $\impliedby$ (2)}
\begin{block}{Claim}
We can assume that the matrix [x'|y'] has a submatrix of the form
\begin{equation*}
\begin{pmatrix}
1 & 1 \\ 1 & -1
\end{pmatrix}
\end{equation*}
\end{block}
\begin{block}{Proof Sketch}
Pivoting A' on entries in N' does not change property (4) so pivot in such a way that the smallest submatrix with its determinant not 0, 1 or -1 is as small as possible. This submatrix has to be of the form of the claim, and a submatrix of [x'|y'].\\
The two rows corresponding to the submatrix have a path in the corresponding graph, thus:
\end{block}
\end{frame}

\begin{frame}
\frametitle{Proof of Tutte's theorem, $\impliedby$ (3)}
A has a submatrix of the form:\\
\begin{center}
$\begin{pmatrix}
1 & 1 & 1 & 0 & 0 & ... & 0 & 0\\
1 & -1 & 0 & 0 & 0 & ... & 0 & 1\\
   &     & 1 & \underline{1} &   &   &   & \\
   &     &    & 1 & \underline{1} & & \mathlarger{0} & \\
* & * &  &  &  1 & \ddots &  &  \\
 &  & & \mathlarger{0} &  & \ddots & \underline{1} & \\
 & & & & & & 1 & 1
\end{pmatrix}$
\end{center}
\end{frame}

\begin{frame}
\frametitle{Proof of Tutte's theorem, $\impliedby$ (4)}
By pivoting on the underlined elements, deleting the rows and columns containing them, multiplying some rows/columns by -1 (and exchanging x' and y' if necessary), we obtain a submatrix of the form:\\
\begin{center}
$\begin{pmatrix}
1 & 1 & 1 & 0 \\
1 & -1 & 0 & 1 \\
a & b & 1 & 1
\end{pmatrix}$
\end{center}
It's still the case that deleting columns x' or y', the remaining matrix is still totally unimodular, so a = 1 and b = 0. In this case, A can be transformed to $M(F_7)$, which contradicts our assumption.
\end{frame}

\section{References}
\begin{frame} 
\frametitle{Original Article}
A Short Proof of Tutte's Characterization of Totally Unimodular Matrices\\
by A. M. H. Gerards\\
Tilburg University, The Netherlands
\end{frame}
 
\end{document}

